\chapter{Implementation}
\label{implementation}

\textit{This section should discuss implementation aspects of the
  large deformation hyperelastic model and the implicit integration
  scheme. I based the Sierra implementation off of the Albany version,
  so that would be a good place to start. In both places we are using
  Sacado to compute derivative for the local consistent tangent, so we
  should talk about what that buys us.}

The details of numerical implementation of the large deformation hyperelastic Gurson model are discussed in this section. A fully implicit integration scheme (\cite{SimoHughes:98, Steinmann1994}) is implemented to integrate stress response over a finite time step $\Delta t = t_{n+1} - t_n$, for given state represented by $\varepsilon_{q(n)}, f_n, \bb_{e(n)}$ and deformation gradient $\bF_{n+1}$. The integration scheme consists of an elastic trial state followed by plastic corrections.

\section{Discrete form of the rate equations}

For the discrete form of the rate equations, the flow rule Eq.\eqref{eq:flowRule} is first written in the material (reference) configuration through a pull-back operation

\begin{equation}\label{eq:flowRule_ref}
-\frac{1}{2}\dot{\bC}_p^{-1}\cdot\bC_p = \gamma\bF^{-1}\frac{\partial\Phi}{\partial\btau}\bF
\end{equation}

Then, the application of the exponential mapping to Eq.\eqref{eq:flowRule_ref} yields an incremental objective integration algorithm

\begin{equation}\label{eq:Cp}
\bC_{p(n+1)}^{-1} = \bF_{n+1}^{-1}\exp\left( -2\Delta\gamma \frac{\partial\Phi_{n+1}}{\partial\btau}\right)\bF_{n+1}\bC_{p(n)}^{-1}
\end{equation}

Applying the push-forward operation to Eq.\eqref{eq:Cp} renders an update algorithm for the elastic left Cauchy-Green tensor

\begin{equation}\label{eq:be_n}
\bb_{e(n+1)} = \exp\left( -2\Delta\gamma\frac{\partial\Phi_{n+1}}{\partial\btau}\right)\cdot\bb_e^{\tr}
\end{equation}

where the trial elastic left Cauchy-Green tensor is given by

\begin{equation}\label{eq:betr}
\bb_e^{\tr} = \bF_{n+1}\cdot\bC_{p(n)}^{-1}\cdot\bF^T_{n+1}
\end{equation}

From elastic and plastic isotropy, $\bb_{e(n+1)}, \bb_e^{\tr}$ and $\btau$ have identical principal axes. Then, the logarithmic Henchy strains follow as

\begin{equation}\label{eq:lnbetr}
\ln\bb_{e(n+1)} = \ln\bb_e^{\tr} - 2\Delta\gamma\frac{\partial\Phi_{n+1}}{\partial\btau}
\end{equation}

From the elastic constitutive relations derived in Eqs.\eqref{eq:p} and \eqref{eq:s}, the Kirchhoff pressure and deviatoric stress tensor can be obtained as

\begin{align}
p_{n+1} &= p^{\tr} - \kappa t\label{eq:discrete_p}\\
\bs_{n+1} &= \bs^{\tr} - 2\mu\Delta\gamma \bn\label{eq:discrete_s}
\end{align}

where $\bn$ and $t$ are given by Eqs.\eqref{eq:n}, \eqref{eq:t} and evaluated at time $t_{n+1}$. The trial state are

\begin{align}
p^{\tr} &= \kappa\ln J_e^{\tr},~~J_e^{\tr} = \det(\bb_e^{\tr})^{1/2}\label{eq:ptr}\\
\bs^{\tr}&= \mu\dev\ln\bb_e^{\tr}\label{eq:str}
\end{align}

The discrete form of evolution equations for internal variables $\varepsilon_q$ and $f$ are obtained by apply backward Euler to Eqs.\eqref{eq:dotf} and \eqref{eq:dotepsilon}. 

\begin{align}
f_{n+1} &= f_n + (1-f_{n+1})t + \sqrt{\frac{2}{3}}\Delta\gamma  k_{\omega}f_{n+1} \omega(\btau) + A_{n+1}(\varepsilon_{q(n+1)} - \varepsilon_{q(n)})\label{eq:discrete_f}\\
\varepsilon_{q(n+1)} &= \varepsilon_{q(n)} + \frac{1}{1-f_{n+1}}\left(\Delta\gamma\sqrt{\frac{2|\psi|}{3}} {\rm sign}(\psi) + \frac{p_{n+1} t}{Y}\right)\label{eq:discrete_eq}
\end{align}

\section{Implicit integration algorithm}
The discrete form of the rate equations \eqref{eq:be_n}, \eqref{eq:discrete_p}, \eqref{eq:discrete_s}, \eqref{eq:discrete_f} and \eqref{eq:discrete_eq} include four unknowns, i.e., the unknown vector
\begin{equation}\label{eq:X}
\bX=\lbrace p, f, \varepsilon_q, \Delta\gamma \rbrace
\end{equation}
which will be obtained from solving the following nonlinear system of equations. For simplicity, in the following section, we will omit the index $n+1$ referring to the current time step. The resulting nonlinear system of equations follow
\begin{align}
R_1(\bX) &= \|\bs^{\tr}\| - 2\mu\Delta\gamma - \sqrt{\frac{2}{3}}{\rm sign}(\psi)\sqrt{|\psi|}Y\label{eq:R1}\\
R_2(\bX) &= p - p^{\tr} + \kappa t\\
R_3(\bX) &= f - f_n - (1-f) t - \sqrt{\frac{2}{3}}\Delta\gamma k_{\omega}f \omega(\btau) - A(\varepsilon_{q} - \varepsilon_{q(n)}) \\
R_4(\bX) &=\varepsilon_q - \varepsilon_{q(n)}- \frac{1}{1-f}\left(\Delta\gamma\sqrt{\frac{2|\psi|}{3}} {\rm sign}(\psi) + \frac{p t}{Y}\right)\label{eq:R4}
\end{align}
The above system of equations can be solved through iterative solution procedures like the Newton's method, which requires consistent linearisations.

The integration algorithm is summarized in the following box\\
Box 1. Integration algorithm for shear-modified finite deformation Gurson model\\
\fbox{\parbox{14.5cm}{
GIVEN: $\varepsilon_{q(n)}, f_n, \bb_{e(n)}$ and $\bF$\\
FIND: $~~~\btau, \varepsilon_{q}, f, \bb_e(\bF_{p}) $\\
STEP 1. Compute trial elastic left Cauchy-Green tensor $\bb_e^{\tr}$ Eq.\eqref{eq:betr}\\
STEP 2. Compute trial Kirchhoff pressure and deviatoric tensor $p^{\tr}, \bs^{\tr}$ Eqs. \eqref{eq:ptr}, \eqref{eq:str}\\
STEP 3. Check yielding Eq. \eqref{eq:Phi}: $\Phi^{\tr}(p^{\tr}, \bs^{\tr}, \varepsilon_{q(n)}, f_n) > 0$ ?\\
$~~~~~~~~~~~~~~$No, set $p = p^{\tr}, \bs = \bs^{\tr}, \bb_e = \bb_e^{\tr}, \varepsilon_{q} = \varepsilon_{q(n)}, f = f_n$  and exit\\
STEP 4. Yes, local Newton loop\\
$~~~~~~~~~~~~~~$4.1 Initialize $\bX^{k}$ Eq. \eqref{eq:X} and iteration count $k=0$\\
$~~~~~~~~~~~~~~$4.2 Assemble residual $\bR(\bX^k)$ Eq. \eqref{eq:R1} - \eqref{eq:R4}\\\
$~~~~~~~~~~~~~~$4.3 Check convergence: $\parallel \bR \parallel < tol$ ?\\
$~~~~~~~~~~~~~~~~~~~$Yes, converged and go to STEP 5\\
$~~~~~~~~~~~~~~$4.4 No, compute local Jacobian matrix $\bJ = \partial\bR / \partial\bX$\\
$~~~~~~~~~~~~~~$4.5 Solve system of equations $\bJ \cdot \delta\bX = \bR$ for $\delta\bX$\\
$~~~~~~~~~~~~~~$4.6 Update $\bX^{k+1} = \bX^{k} - \delta\bX$, $k\rightarrow k+1$ and go to 4.2\\
STEP 5. Update $\btau = \bs + p\bg$, and $\varepsilon_q, f, \bF_p^*$\\
}}

$^*$The plastic deformation gradient, which is used in \eqref{eq:flowRule_ref} and \eqref{eq:betr} to compute trial state, is updated using
\begin{equation}
\bF_p = \exp\left( \frac{\partial\Phi}{\partial\btau}\right)\cdot\bF_{b(n)}
\end{equation}

The linearization of the system of equations requires evaluating the local Jacobian matrix $\bJ = \partial\bR / \partial\bX$. In this work, we used an numerical exact method called Fast Automatic Differentiation (FAD) to evaluating all the Jacobian. FAD provides an efficient and convenient way to evaluate derivatives. A detailed description of FAD can be found in [CITE].

\section{FAD: a numerical exact way of computing consistent tangent}


% Local Variables:
% TeX-master: "GursonModel"
% mode: latex
% mode: flyspell
% End:
